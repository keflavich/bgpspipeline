% calibration
% you should look @ calibration/plot_dcfluxes.pro in my publish/ directory, and /usb/scratch1/planets/planet_dcfluxes.txt is the actual raw data
%/home/milkyway/student/ginsbura/bgps_pipeline/calibration/plot_dcfluxes.pro?

% glitches
% I can't send you raw data for the glitches, but the best way I can think of dealing with them is taking one of my postiter.sav files in some directory and grabbing ac_bolos[where(flags)]...... except that include hand-flagged stuff too

%%%%%%%%%%%%%%%%%%%%%%%%%%%%%%%%%%%%%%%%%%%%%%%%%%%%%%%%%
%                                                       % 
%       Bolocam Galactic Plane Survey Description       %
%                                                       % 
%%%%%%%%%%%%%%%%%%%%%%%%%%%%%%%%%%%%%%%%%%%%%%%%%%%%%%%%%

\documentclass[12pt,preprint]{aastex}
%{emulateapj}

\usepackage{rotating}

\citestyle{aa}

\bibliographystyle{apj_w_etal}

\newcommand{\vect}[1]{\mathbf{#1}}
\newcommand{\xad}{\vect{x}}

\newcommand{\vdag}{(v)^\dagger}
\newcommand{\myemail}{jaguirre@nrao.edu}
\newcommand{\lsim}{{_{<}\atop^{\sim}}}
\newcommand{\gsim}{{_{>}\atop^{\sim}}}
\newcommand{\etal}{{et al.\/}}
\newcommand{\ie}{{\em ie.\/}}
\newcommand{\cmq}{cm{$^{-3}$}}
\newcommand{\per}{$^{\rm{-1}}$}
\newcommand{\tc}{{$\theta^1$~Orionis~C}}
%\newcommand{\msol}{M{$_{\odot}$}}
\def\msol{\ifmmode {\>M_\odot}\else {$M_\odot$}\fi}
\newcommand{\lsol}{L{$_{\odot}$}}
\newcommand{\kms}{km~s{$^{-1}$}}
\newcommand{\hii}{H~{\sc ii}}
\newcommand{\Hii}{H~{\sc ii}}
\newcommand{\Ha}{\mbox{H$\alpha$}}
\newcommand{\sii}{S~{\sc ii}}
\newcommand{\Feii}{Fe~{\sc ii}}
\newcommand{\oi}{O~{\sc i}}
\newcommand{\nii}{N~{\sc ii}}
\newcommand{\oiii}{O~{\sc iii}}
\newcommand{\mgii}{Mg~{\sc ii}}
\newcommand{\tco}{{$^{13}$CO}}
\newcommand{\CO}{{$^{12}$CO}}
\newcommand{\Tco}{{$^{12}$CO}}
\newcommand{\co}{C{$^{18}$O}}
\newcommand{\Lsol}{L$_{\odot}$}
\newcommand{\Msol}{M$_{\odot}$}
%\newcommand{\C18o}{C$^{18}$O($1\rightarrow 0$)}
\newcommand{\Check}{{\bf ???}}
\newcommand{\mum}{\ensuremath{\mu \mathrm{m}}}
\newcommand{\flux}{flux density}
\newcommand{\solar}{\ensuremath{\odot}}

\newcommand{\epsi}{\varepsilon}

\newcommand{\bgpsarea}{150}
\newcommand{\bcamfwhm}{31.2\arcsec}
\newcommand{\ncores}{$10^4$}
\newcommand{\bgpsdepthlow}{20}
\newcommand{\bgpsdepthhigh}{50}

\newcommand{\TBD}{{\bf TBD}}

\def\Figure#1#2#3#4{
\begin{figure}[htb]
\epsscale{#4}
\plotone{#1}
\caption{#2}
\label{#3}
\end{figure}
}

\def\Table#1#2#3#4#5{
\begin{deluxetable}{#1}
\tablewidth{0pt}
\tablecaption{#2}
\tablehead{#3}
\startdata
\label{#4}
#5
\enddata
\end{deluxetable}
}

\newcommand{\penn}{1}
\newcommand{\casa}{2}
\newcommand{\cso}{3}
\newcommand{\utexas}{4}
\newcommand{\virginia}{5}
\newcommand{\ubc}

	% End definitions

\slugcomment{DRAFT: \today}

\shorttitle{BGPS}
\shortauthors{Aguirre et al.}

\begin{document}

\title{The Bolocam Galactic Plane Survey: Survey Description and Data
Reduction}

%\newpage
%\pagebreak

%\tableofcontents

%\newpage
%\pagebreak

\author{James Aguirre\altaffilmark{\penn},
        Adam Ginsburg\altaffilmark{\casa},
        Miranda Nordhaus\altaffilmark{\utexas},
	Meredith Drosback\altaffilmark{4},
        John Bally\altaffilmark{\casa}, 
	Cara Battersby\altaffilmark{\casa},
	Eric Todd Bradley\altaffilmark{5},
%        Richard Chamberlin\altaffilmark{\cso},
	Claudia Cyganowski\altaffilmark{6},
	Darren Dowell\altaffilmark{7}
	Neal J. Evans II\altaffilmark{\utexas},
        Jason Glenn\altaffilmark{\casa},
        Paul Harvey\altaffilmark{1,7},
        Erik Rosolowsky\altaffilmark{8},
%        Wayne Schlingman\altaffilmark{9},
%        Yancy Shirley\altaffilmark{9},
        Guy S. Stringfellow\altaffilmark{\casa},
        Josh Walawender\altaffilmark{10}, and 
        Jonathan Williams\altaffilmark{11}
}


%\author{James Aguirre\altaffilmark{\penn},
%  Adam Ginsburg\altaffilmark{\casa},
%  John Bally\altaffilmark{\casa},

%  Meredith Drosback\altaffilmark{\virginia},
%  Neal J. Evans\altaffilmark{\utexas},
%  Jason Glenn\altaffilmark{\casa},
%  Miranda Nordhaus\altaffilmark{\utexas},
%  Erik Rosolowsky\altaffilmark{\ubc},
%  Josh Walawender\altaffilmark{\virginia},
%  Jonathan Williams\altaffilmark{\virginia}
%}


\email{jaguirre@sas.upenn.edu}

\altaffiltext{\penn}{University of Pennsylvania, Philadelphia, PA 19104}
%Jansky Fellow, National Radio Astronomy Observatory} 
\altaffiltext{\casa}{CASA, University of Colorado, CB 389, Boulder, CO 80309}
\altaffiltext{\utexas}{Department of Astronomy, University of Texas, 
       1 University Station C1400, Austin, TX 78712}
\altaffiltext{4}{Caltech Submillimeter Observatory, Hilo, HI}
\altaffiltext{5}{Institute for Astronomy (IfA), University of Hawaii
640 N. Aohoku Pl., Hilo, HI 96720}

\begin{abstract}

We present the Bolocam Galactic Plane Survey (BGPS), a 1.1 mm
continuum survey of \bgpsarea\ square degrees of the Galactic Plane in
the 1st, 2nd, and 3rd quadrants.  The BGPS represents the first large
area, systematic survey of the Galactic Plane in the (sub)millimeter
continuum without pre-selected targets.  The survey has detected
\ncores\ cores to a limiting depth of between \bgpsdepthlow\ and
\bgpsdepthhigh\ mJy RMS.  This paper details the survey observations
and methods.  We treat carefully the comparison of the survey pointing
and flux calibration and and serves as a companion to the public data
release through NASA's Infrared Processing and Analysis Center (IPAC).
We anticipate that the BGPS will be useful as a finder chart for ALMA
and as a long-wavelength complement to {\em Herschel}-HiGAL.

\end{abstract}

\keywords{
ISM: - molecular clouds --
stars: formation -- high mass
millimeter continuum
}

%\include{bgps_methods_introduction}

\section{Introduction}

Millimeter-wavelength continuum surveys of the Galactic plane provide
the best way to identify high-column density cores and clumps where
planets, stars, and star clusters form.  Such data can locate and lead
to the measurement of the physical properties of cores un-biased by
selection effects such as the presence of embedded stars, star
clusters, infrared sources, masers, or radio continuum emission.
Galaxy-wide surveys are essential for measuring the impacts of the
environment on core properties and star formation activity.  Do core
properties vary with location in or out of spiral arms?  Do core
properties vary with Galactocentric distance?  Do they depend on the
level of nearby star formation activity?  Galactic plane surveys of
mm-wavelength dust continuum emission provide the most efficient tool
for the identification of potential or active sites of star formation,
including the rare objects where the most massive stars and cluster
are, or soon will be, forming.  Dust continuum surveys of the Galaxy
provide essential ``ground truth'' required for the analysis of distant
galaxies where where individual clouds and cores are not resolved and
only galaxy-wide average quantities can be measured.

The advent of focal plane arrays containing hundreds of individual
bolometers sensitive to millimeter and sub-millimeter (sub-mm)
radiation has enabled large-scale surveys of mm-wavelength continuum
emission from Galactic interstellar dust.  The 144 element Bolocam
focal plane array, mounted at the Cassegrain focus of the 10 meter
diameter sub-mm telescope at the Caltech Sub-millimeter Observatory
(CSO) was use to survey over \bgpsarea\ square degrees of the northern
Galactic plane at a wavelength of 1.1 mm in the dust-continuum.  The
Bolocam Galactic Plane Survey (BGPS) presented here has detected about
\ncores\ cores in the surveyed fields, providing an unbiased sample of
cores suitable for multi-wavelength and high-resolution studies with
existing telescopes and future facilities.  The core catalog is
described in a companion paper \citep{rosolowsky09}.

Dust continuum emission near a wavelength of 1 mm is the best tracer
of the material most directly associated with the formation of
planetary systems, stars, and star clusters
\citep{johnstone06}.  Galactic dust continuum radiation at
this wavelength is optically-thin almost everywhere, and at typical
dust temperatures of order 10 K or more, is observed on the
Rayleigh-Jeans tail of the Planck function.  Thus, the determination
of column densities is relatively straight forward if the dust
temperature is known.

A crucial step in the observational study of star and cluster
formation is identification and characterization of the cloud cores
which will soon form or are actively forming stars.  Massive stars and
clusters form from high-density cloud cores with very large column
densities and extinctions $A_V > $ 100.  Thus, such cloud cores are
best investigated at millimeter and sub-millimeter wavelengths.
However, the interpretation of various gas tracers that produce
emission lines in this portion of the spectrum is very difficult.
Although in principle, spectral lines provide excellent diagnostics of
line-of-sight motions, temperatures, and densities in a cloud,
variations in tracer abundances caused by depletions and complex
chemical processing, uncertainties in excitation conditions, and the
impacts of radiation fields and shocks make the derivation of column
densities, masses, and other physical properties of cloud cores very
difficult.  The continuum emission from warm dust provides a more
reliable tracer of the column densities and core masses.

The great advantage of continuum observations at 1.1mm is that the
mass column density does not depend strongly on the dust temperature
$T_d$ (varying only linearly unless $T_d$ is less than about 10 K) and
is independent of the gas volume density.  Using the same dust opacity
as \citet{enoch06} and \citet{young06}, and assuming a gas-to-dust
mass ratio $X$ of 100, the mass sensitivity can be written as
\begin{equation} 
\label{eq:Mass}
M_{gas}\approx
0.22\left( e^{12.9/T_d}-1 \right) \left({S_\nu\over 15\; {\rm mJy}} \right)
D^{2}_{kpc} \msol 
\end{equation}
where $D_{kpc}$ is the distance to the source in kpc.  (At $S_\nu =
15$~mJy this corresponds to an RMS extinction sensitivity of $A_V\sim
1$ mag, with the same $T_d$ dependence).  It is thus possible with a
wide area, relatively shallow 1.1~mm survey to provide an unmatched,
uniform inventory of massive star-forming and starless condensations.

A number of (sub)millimeter surveys of the Galactic Plane are ongoing
or planned.  The BGPS is the first.

The outline of the paper is as follows. Section \ref{sec:Observations}
describes the instrument and the observations.  Section
\ref{sec:FluxCalibration} describes the flux calibration and errors.
Section \ref{sec:Astrometry} describes the construction of an absolute
reference frame for the observations and verification against 

%BGPS will identify regions of high-mass star and cluster formation
%throughout the Galaxy and sites of low-mass star formation within a
%few kpc of the Sun (Figure 1).  Their physical properties may then be
%constrained by comparison with data in other wavebands.  The
%distribution of these regions will be compared with other tracers of
%the interstellar medium (ISM) such as molecular clouds, supershells,
%HII regions, supernova remnants, and supershells.  The 1.1 mm data
%together with dust continuum maps at shorter wavelengths taken by us
%with SHARC~II or with other telescopes will constrain temperature and
%reveal what surrounding material may be feeding the formation of dense
%cores. Star formation efficiencies may be obtained by detailed
%comparison of the NIR-MIR observations of stars with the BGPS
%observations.  This will also constrain the extinction law and dust
%properties, essential to refining mass estimates.  By comparing the
%distribution of 1.1mm cores with HII regions, supernova remnants, and
%IR emission as seen by {\it Spitzer}, we will constrain the mode of
%core and star formation, i.e., is it predominantly triggered by
%advancing shock and ionization fronts, by converging flows in a
%turbulent medium, or by spontaneous gravitational collapse?  The
%former models predict a close correlation between fronts and cores
%while the latter predicts the correlation will be poor.


%\Figure{mass_limits}{The limiting gas mass of cores detectable
%with at a flux limit of 15 mJy RMS. Detection thresholds at 3 (dashed)
%and 5 (solid) $\sigma$ are shown for a range of dust temperatures.
%Dust temperatures $<20$~K are most interesting for this survey, since
%these are the sources most likely to be pre- or proto-stellar
%condensations.}{fig:mass_limits}{1.0}
%
\Figure{coverage_moll} {The coverage of the BGPS.  The background
greyscale is IRAS 100 \mum.  
%The Bolocam coverage is shown with white
%solid lines in the top two panels and black in the bottom three.
%Major regions of Galactic star formation are labeled.  The coverage of
%{\em Spitzer}-GLIMPSE \citep{benjamin03} and the FCRAO $^{13}$CO GRS
%\citep{jackson06} are shown.
}{fig:coverage}{1.0}

%\include{bgps_methods_observations}
\section{Observations}
\label{sec:Observations}

%\subsection{Instrument}

Bolocam\footnote{{\tt http://www.cso.caltech.edu/bolocam}} is the
facility 144-element bolometer array camera operating from the 10.4 m
dish of the Caltech Submillimeter Observatory (CSO) on the summit of
Mauna Kea.  We used the filter configuration with a bandcenter of
268~GHz (hereafter 1.1~mm) and fractional bandwidth $\Delta \nu/\nu =
0.17$.  The passband is constructed to exclude the $^{12}\mathrm{CO}(2
\to 1)$ emission line.  Figure \ref{fig:Passband} shows the passband.
We compute color corrections for sources of varying spectral index in
Appendix \ref{app:ColorCorrections}.

The array field-of-view is 7\arcmin.5, with individual detectors
having nearly Gaussian beams of \bcamfwhm\ FWHM.  The spacing of the
pixels at 1.1 mm is 1.6$f\lambda$, so the focal plane is not
instantaneously sampled.  The Bolocam instrument is described in greater
detail in \citet{haig04} and \citet{glenn03}.

%\subsection{Observations}

The observations described here were acquired during six separate
observing sessions at the CSO over the course of two years.  The
observing epochs are given in Table \ref{tab:Observing}, along with
corresponding ranges for the zenith opacity $\tau_{225}$ of the CSO
tipper tau.  Because of the CSO weather multiplexing policy, Bolocam
observations were typically scheduled when $\tau_{225} > 0.06$.
Between each observing epoch, Bolocam was removed from its mount at
the re-imaged Cassegrain focus and stored warm.  Thus the flux
calibration and pointing model must be re-computed for each epoch to
allow for changes in the instrument and optics.

%\subsection{Observing Strategy}

% xraster 120 11280 23 /step_size=162 /equatorial /position_angle=36.753118 /alternate_direction /rot /settling_time = 03

Our basic observing strategy was to raster scan Bolocam by moving the
primary mirror of the CSO to modulate the astrophysical signal faster
than fluctuations in atmospheric opacity.  Each field was scanned with
alternating raster scans along $l$ and $b$.  Starting with Epoch III,
the fundamental scan unit was a $3\arcdeg \times 1\arcdeg$ block.
Each block was scanned with 23 scans along lines of constant $b$, and
67 scans along lines of constant $l$.  In both cases, the spacing
between adjacent rasters was 162\arcsec.  The total time for such a
scan was between 39 and 48 minutes.  In order to fill in the
instanteously undersampled focal plane, we began using in Epoch II a
field rotator to mechanically align the focal plane at an optimal
orientation along the scan direction to improve the sampling
orthogonal to the scan direction.  This is shown in Figure
\ref{fig:Array}.  The data were electronically sampled at 10 Hz along
the scan direction, slightly higher than the Nyquist rate for the scan
speed of 120\arcsec\ s$^{-1}$.  
%CHECK
Without the field rotator, the coverage shows variations of 100\% from
pixel to pixel in a single raster of a field.  With the rotator

The resulting coverage using the
above scan pattern and the field rotator, is typically uniform to
% per Adam, 2009/3/26

 with only a single pass across the field.

% FIGURES AND TABLES

\Figure{bolocam_bandpass}{The Bolocam 1.1 mm bandpass.  Bright
molecular emission lines are shown at their rest frequencies.  Note
that the Bolocam passband rejects $>90\%$ of the $^{12}$CO flux, leaving our
continuum measurements largely uncontaminated.  SO and CH3OH lines are probably
the dominant contributors to line flux in the passband.  \citet{nummelin1998}
found that 22\% of the  flux in one pointing towards Sgr B2 was from lines, and
\citet{yoshida2005} reported >40\% in Orion A from line emission}
{fig:Passband}{1.0}

\Table
{llll}
{Observing Epochs for the BGPS}
{Begin (UT) & End (UT) & Nights & $\tau_{225}$ }
{tab:Observing}
{
2005 Jul 03 & 2005 Jul 13 & 5 & 0.1 \\
2005 Sep 04 & 2005 Sep 12 & 5 & 0.1 \\
2006 Jul 01 & 2006 Jul 13 & 5 & 0.1 \\
2006 Sep 01 & 2006 Sep 09 & 5 & 0.1 \\
2007 Jul 01 & 2007 Jul 13 & 5 & 0.1 \\
2007 Sep 01 & 2007 Sep 09 & 5 & 0.1 \\
}

\Figure{scan_10deg}{The effect of field rotation on the coverage
obtained via Bolocam raster scans.  (could also use this figure to
show the distortion map)}{fig:Array}{1.0}

%\begin{sidewaysfigure}[h]
\begin{center}
  \begin{figure}[t]
%    \includegraphics[width=6in,height=9.5in]{fig1_pg1_bw_inv_bar_label} 
    \includegraphics[scale=0.9]{fig1_pg1_bw_inv_bar_label}
    \caption{$l=-10.5$ to $l=19.5$ Units are Jy/Beam.  The brightest
      sources, e.g. Sgr B2, Sgr A, and sources near $l=10$ and
      $l=13$, appear to be saturated, but this is only a display
      artifact.  The noise is more pronounced from $l=-7$ to $l=-2$
      because this region was observed less.}
    \label{fig:galplane_p1}
  \end{figure}
    %\end{sidewaysfigure}

    %\begin{sidewaysfigure}[h]
  \begin{figure} 
%    \includegraphics[width=6in,height=9.5in]{fig1_pg2_bw_inv_bar_label} 
    \includegraphics[scale=0.9]{fig1_pg2_bw_inv_bar_label} 
    \caption{$l=19.5$ to $l=49.5$.  Units are Jy/Beam .  G34.3+0.15, W
      51, W 43, W 49, and M 17 appear to be saturated, but this is
      only a display artifact}
    \label{fig:galplane_p2}
    %\end{sidewaysfigure}
  \end{figure}
\end{center}


  

%\begin{sidewaysfigure}[h]
%\begin{center}
%\includegraphics[scale=0.5]{bgps_montage}
%\end{center}
%\caption{ The coverage of the Bolocam Northern Galactic Plane Survey.
%as of September 2007.  The negative greyscale is the IRAS 100 \mum\
%survey.  Dashed lines indicate coverage between 15 and 100 mJy RMS;
%solid lines indicate areas covered more deeply than 30 mJy RMS.}
%\label{fig:coverage_sep06}
%\end{sidewaysfigure}





%\include{bgps_methods_flux_calibration}
\section{Flux Density and Surface Brightness Calibration}

The absolute flux calibration is obtained from observations of Mars,
Uranus and Neptune (the ``primary calibrators'').  The millimeter-wave
flux of these planets is known to $\sim 5\%$
\citep{orton86,griffin93}.  Calibrations at (sub)millimeter
wavelengths are strongly affected by atmospheric opacity corrections.
Further, the detector responsivity of Bolocam's bolometers is a
non-linear function of the mean atmospheric loading.

To relate observations of the primary calibrators to observations of
the BGPS fields, we make use of the following relation.  The
calibration ${\cal S}$, referred to the detectors, is given by
\begin{equation}
\label{eq:Calibration}
{\cal S} \, \left[\mathrm{\frac{V}{Jy}}\right] = 
S(\tau) \eta A \exp{(-\tau)} \Delta\nu
\end{equation}
where $S$ is the bolometer responsivity ([V/W]), $\eta$ is the system
optical efficiency, $A$ is the effective telescope collecting area,
$\Delta \nu$ is the bandwidth, and $\tau$ is the line-of-sight,
in-band atmosphere opacity.  Under the assumption that the only power
variation on the detectors is due to the power from the atmosphere,
which may be parameterized by $\tau$, ${\cal S}$ is a single-valued
function of $\tau$.  We have used the mean bolometer resistance as a
proxy for $\tau$, since the bolometer resistance is a single-valued
function of loading.  {\it This quantity is monitored continuously for
all observations.}  Note that this calibration curve folds in both the
effects of changing atmospheric transmission, as well as changes in
the detector response with optical loading.

%Multiple observations of secondary calibrators suffice to give the
%shape of the curve.  To reference it to the absolute calibrators, the
%fluxes of the planets were held fixed and the fluxes of the secondary
%calibrators allowed to float to produce the best agreement with a
%second-order polynomial for all sources.  The values for the secondary
%calibrators agree well with previously obtained Bolocam values and
%with values in the literature \citep{sandell94,jenness02}.  
%%A fit of Equation \ref{eq:Calibration} 
%

A fit to the observed values for the primary calibrators was performed
for the each epoch separately.
%CHECK
The agreement between epochs was good, so a single combined
calibration was used for all data here.
%We have found that the calibration can differ
%between successive cooldowns of the instrument by $\sim10\%$ \Check\
%but is generally stable over the period of order a month which
%constitutes each of the observing epochs.  
The calibration curve is shown in Figure \ref{fig:Calibration}.  The
resulting error on the calibration curve fit is less than 
%CHECK
$5\%$ (statistical) over the observed range of $\tau$.  A tally of all
contributions to the flux density error is given in Table
\ref{tab:ErrorBudget}.

The above flux density calibration only accounts for the average
calibration of bolometer Volts to Janskys.  It is also necessary to
account for the variation of bolometer response across the focal
plane.  We do this by monitoring the response to the atmosphere
emission in all bolometers.  Since the atmosphere is common to all
detectors, any changes in response are due to the intrinsic properties
of the detectors.  The change in relative response with loading for a
typical detector is shown in Figure \ref{fig:Flatfield}.  The method
is discussed further in Section \ref{sec:Mapping}



The instrument beam size (PSF) is measured using the planets Uranus
and Neptune which are nearly point sources for Bolocam.
% For 


%The semi-diameter of
%Mars over this period ranged from 2.3\arcsec\ to 2.9\arcsec, with
%Neptune $\sim1.1\arcsec$.  Relative to the \bcamfwhm~FWHM Bolocam
%beam, these may be regarded as point sources; the finite size of Mars
%results in a $< 1.5\%$ correction of the observed flux.
The beam area is that averaged over all detectors, and is ?$1.71\times10^{-6}$\TBD\ ($35.6$\farcs)
steradians.  This allows conversion of the maps, calibrated in Jy
beam$^{-1}$, into surface brightness (MJy sr$^{-1}$).  The uncertainty
in the beam size is ?$.7\%$\TBD\  (number quoted is the uncertainty in the 
mean beam size), leading to an uncertainty in the surface
brightness calibration of \TBD.

\subsection{Comparison to Other Surveys}

Comparison of the accuracy of the Bolocam flux density calibration
with the SCUBA 850 \mum\ results is complicated by the unknown effect
of the dust spectral index.  However, there are several large-area
surveys with the MAMBO instrument at 1.2 mm with which we can compare.
For example, there is \citet{motte07}.

\Figure{fluxcal_errorbars}
%{calibration_plot_2004}
{Average calibration curve [V/Jy] versus the mean detector voltage, a
proxy for atmospheric loading.}{fig:Calibration}{1.0}

\Figure{rel_flux_cal_050303_ob3_to_050405_o66}{Scaling of the
relative calibration with atmospheric loading for bolocam
143. {\em To be replaced with new figure}}{fig:Flatfield}{1.0}

\Figure{relsens_cal}{Relative sensitivity calibration.  Black is the median over
all bolometers, red and green are individual bolometers. a. before
b. after}{fig:relsens_cal}{1.0}

\begin{deluxetable}{lc}
\tablewidth{0pt}
\tablecaption{Flux density error budget}
\tablehead{
\colhead{Source} & \colhead{Contribution}
}
\startdata
\label{tab:ErrorBudget}
Pointing error uncertainty & 2\% \\
Calibration curve (Eq. \ref{eq:Calibration}) uncertainty & $<5\%$ \\  % formal statistical error is probably less than real effect...?
PCA flux reduction uncertainty & 4\% \\   % is this an uncertainty?  Or systematic loss?
Absolute Mars flux uncertainty &  5\% \\
Beam size uncertainty (surface brightness) & X\% \\
\hline
Total & 7\% (random) + 5\% (systematic)
\enddata
\end{deluxetable}

\section{Astrometry}

\subsection{Absolute Reference Sources and Pointing Model}

We constructed an absolute reference system for the BGPS by observing
bright quasars and blazars near the Galactic Plane.

The distribution of sources over the sky is shown in Figure \ref{fig:astrometry}

Pointing observations were performed approximately once every two
hours over the night.  These sources were then mapped in alt/az
coordinates relative to the nominal telescope boresight.  The
elevation offsets showed a deviation from zero which was empirically
well-modeled by a quadratic function of elevation and a linear
function of azimuth.  No systematic deviation was observed for the
azimuth offsets.  

Because the Galactic Plane appears over much of the sky, the pointing
model necessarily encompasses a large range of elevation and azimuth.
Nevertheless, an RMS scatter of $\sim 6\arcsec$ for the model over the
entire observed range was achieved, with the scatter somewhat worse at
the highest ZA.  A separate model was constructed for each epoch.  The
residuals are nearly Gaussian, with an RMS of $\sim6\arcsec$.  

Further details of the pointing calculation are given in Appendix
\ref{sec:PointingCalculation}.

\subsection{Pixel Positions in the FOV}

In addition to the model for the pointing center, it is necessary to
empirically determine actual projected pattern of the array on the
sky, and measure the offset angle between the focal plane and the sky
coordinate system.  This is done by making observations which track
all detectors across a bright source (a planet) and making maps from each
bolometer individually.

A mean focus offset for the secondary is determined once per observing
run.  An elevation-dependent correction is applied via a
CSO-determined look-up table between raster scans.  We do not see any
evidence that more frequent focus checks improve the quality of the
PSF.

\subsection{Relative Alignment and Mosaicing}

Relative alignment was performed by finding the peak of the
cross-correlation between images and a pointing master selected from
the epoch with the best-constrained pointing model for that
field. Each observation was initially mapped individually, then all
observations of a given field were cross-correlated with a selected
master image of that field. The cross-correlation peak was fit with a
gaussian and the difference between the gaussian peak and the image
center was used as the pixel offset. The offsets were recorded and
written to the timestreams. Finally, all observations of a field were
merged into a single timestream with pointing offsets applied to
create the field mosaic.

This method of alignment avoids the ambiguities inherent in using
extracted sources to align fields, as the Bolocam sources are rarely
point-like.  It further avoids the pointing smearing and loss of peak
flux density which would result if the the maps were coadded using the
pointing model alone.

TBD: In low s/n fields, specifically l=65 to 75, there was not enough signal to
acquire a pointing offset using cross-correlation.  In these fields the
relative pointing offset is larger and the beam is slightly larger.

{\em Show a cross-correlation map?}

\subsection{Comparison to SCUBA and SHARC-II Positions}

The same cross-correlation procedure was used to compare to SCUBA maps where
available.  The morphological correlation between SCUBA and BGPS sources is
excellent, so cross-correlation maps could be used to test our pointing model.

{\em Miranda's data here}

\Figure{plane_astrometry}{The distribution of absolute reference
sources (crosses) and in-plane millimeter sources (x's) in the
Galactic Plane.}{fig:astrometry}{1.0}

\Figure{pointing_model}{The pointing model correction (left) and
residuals (right) for Epoch V, from which the master reference images
were derived for subsequent alignment. {\em To be replaced with newer
figure?}}{fig:PointingModel}{1.0}

\Figure{pointing_model_residuals}{The residuals of the pointing
model.  Note the Gaussian distribution.}{fig:PointingModelResiduals}{1.0}

\Figure{pointing_model_skydist}{The distribution of the PCAL
sources across the local sky in Hawaii.  Notice the good sampling of
the entire region used for the Survey.}{fig:PCALSkydist}{1.0}

\section{Mapping Algorithm}
\label{sec:Mapping}

The essential signal-processing problem to be solved by the mapping
algorithm is the estimation and removal of a signal due to fluctuating
atmosphere emission which is $>\sim$100's of times stronger than the typical
sources in our maps.  We implement an iterative procedure for
estimating both the atmosphere fluctuations and the astrophysical
signal without {\it a priori} knowledge of either. 

We assume the raw timestream data $d$ for each bolometer (indexed by
$i$) can be written as
\begin{equation}
d_i(t) = s_i(t)+a_i(t)+e_i(t)+\epsi_i(t)
\end{equation}
where $s$ is the astrophysical signal, $a$ is atmospheric fluctuation
noise, $e$ are non-random signals due to the instrument itself, and
$\epsi$ is irreducible Gaussian noise due to photon shot and detector
noise.  The process of making a maximum likelihood map in the presence
of Gaussian noise is well-understood, and if the only contributions to
the data were $s$ and $\epsi$, then the minimum variance estimator of
the true astrophysical map is given by
\begin{equation}
\label{eq:LSMap}
m = (A^T W A) A^T W d
\end{equation}
where $A$ encodes the pointing information and $W$ is the covariance
matrix of the noise $\epsi$.  Note that the mapping from data to map
is a linear operator, $M =(A^T W A) A^T W$.  For compactness, Equation
\ref{eq:LSMap} will be written as $M[d] = m$.  Note that the mapping
operation is not invertible.  However, note that given a map $m$,
there exists a linear operation which makes predictions about the
observed timestreams, namely $d = A m$; this will be denoted $T[m] =
d$.

The goal is to produce a time series for each bolometer which as
closely as possible approximates $s + \epsi$ so that we may produce
the best estimate of the astrophysical signal $m = M[s+\epsi] = S + N$.

We proceed iteratively as follows. 
\begin{equation}
\label{eq:FirstIterStep}
\tilde{a}^{(n)}(t) = 
\sum_i{d_i(t)-\tilde{s}^{(n)}_i(t) - \tilde{e}^{(n)}_i(t)}
\end{equation}
where $\tilde{s}^{(0)}_i(t)=0$.  This mean atmosphere model is then
fit to each bolometer to obtain the relative gains (flat field) of the
detectors
\begin{equation}
\tilde{a}^{(n)}_i(t) = \frac{d_i(t) \cdot a(t)}{d_i(t)^2} a(t) = r_i a(t)
\end{equation}
The instrument errors, which are typically smaller than the
atmosphere, are then estimated
\begin{equation}
\tilde{e}^{(n)}_i(t) = some models
\end{equation}
and both the atmosphere and instrument models are subtracted from the
raw time series
\begin{equation}
d_i(t) - \tilde{a}^{(n)}_i(t) - \tilde{e}^{(n)}_i(t)
\approx s_i(t) + \epsi_i(t)
\end{equation}
This is the best estimate of signal plus irreducible noise at
iteration step $n$, and is made into a map
\begin{equation}
M[d_i(t) - \tilde{a}^{(n)}_i(t) - \tilde{e}^{(n)}_i(t)] = \tilde{m}^{(n)}
\end{equation}
The current best map $\tilde{m}^{(n)}$ is then deconvolved to provide
a relatively low-noise, smooth map from which to generate a timestream
\begin{equation}
T[{\cal D}[\tilde{m}^{(n)}]] = \tilde{s}^{(n)}_i(t)
\end{equation}
We note that it is possible to produce an ``error map''
\begin{equation}
N = M[d_i(t) - \tilde{s}^{(n)}_i(t) - \tilde{a}^{(n)}_i(t) - 
\tilde{e}^{(n)}_i(t)]
\end{equation}
The process then begins again with Equation \ref{eq:FirstIterStep}.

%\Figure{eachpca}{The effect of the number of iterations on
%the reduction process.}{fig:Iter}{1.0}

\subsection{Atmosphere Fluctuation Noise Model}

The simplest model for the atmosphere fluctations makes use of the
fact that the beams for all of the detectors pass through a nearly
identical atmosphere, thus implying that atmosphere fluctuations will
be highly correlated between detectors.

\subsection{Instrument Error Signals}

Most of the instrument error signals have characteristic features
which allow them to be identified, though not always removed.  The
error signals include the following:
\begin{enumerate}

\item Pickup from the 60 Hz AC power.  This appears as narrow lines in
the PSD of the data.  The second harmonic of 60 Hz is aliased via
beating against the 130 Hz bolometer bias frequency into 10 Hz with
sidebands split at $\sim1$ Hz.

\item Spikes in voltage due to cosmic ray strikes on the bolometers
(``glitches'').

\item Microphonic pickup due to vibrations of the receiver.  The most
noticeable microphonics occur at the end of each scan when the
telescope is turning around and the field rotator is adjusting.  This
leads to broad spikes in the time series, whose long decay must be
removed from the data, particularly during the beginnings of scans.

%\item Others?

\end{enumerate}

Fortunately, most of the AC powerline pick-up occurs at frequencies
where there is no astrophysical signal, given the beam size and scan
speed.  Thus this error is dealt with by first notch filtering at the
line frequencies and then low-pass filtering the data.  Because of the
low correlation with astrophysical signal, this step is performed only
once and is not iterated.

Both glitches and microphonic pickup from the scan turnarounds are
degenerate with astrophysical signal and must be estimated as part of
the iterative process.  Glitches are identified as large excursions
from the RMS level after subtraction of the best atmosphere and bright
source model, and the data there is excluded from inclusion in
subsequent maps.  The turnaround microphonics are modeled as decaying
exponentials at the beginnings and ends of scans.

\Figure{flagger_plots}{An illustration of a ``glitch'' in a single
bolometer timestream due to a cosmic ray strike.  Note the acausal
ringing due to the application of the downsampling filter.  The time
series for a physically adjacent bolometer is shown in red.  The high
level of correlation due to atmospheric fluctuations is readily
apparent, as well as the lack of correlation due to the
glitch.}{fig:CosmicRayGlitch}{1.0}

\Figure{crs}{The distribution of glitch amplitudes flagged and
removed from the data.  The expected contribution to the noise from
the unremoved glitches below the detection threshold is
\TBD.}{fig:GlitchDistribution}{1.0}

\subsection{Data Flagging}

Due to the large volume of data generated by the survey, it was
necessary to develop new tools to quickly visualize the data and
ensure data quality.  We used a ``waterfall''

\Figure{flagger_withglitch}{An illustration of the flagging process using the
waterfall plot.  In this image, an anomaly which affects all
bolometers for a brief period of time is evident near the top of the
image.}{fig:Flag1}{1.0} 

\Figure{flagger_glitchboxflagged}{A box is drawn around the bad data and it is
removed from subsequent analysis.}{fig:Flag2}{1.0} 

\Figure{flagger_glitchgone}{A rescaled version of the data with the bad data
removed.}{fig:Flag3}{1.0}


An automated flagger was also created that flags out outlier data on a per-pixel
basis.  In order to make a robust measurement of the variance of the fluxes
assigned to each pixel, we used the median average deviation over the data in
the pixel and rejected high and low outliers at the 3-sigma level.  Pixels with
too little data to compute a deviation, i.e. those with $<3$ data points, were
also flagged out - these scan-edge pixels are the dominant contribution to the
total number of flagged data points.

\subsection{Creation of the Astrophysical Model}

WRITTEN POORLY: 
The timestream data is made into a spatial map using the pointing data 
corresponding to each time point for each bolometer.  The data is weighted
by inverse variance across a single scan and then drizzled into a map with
$7.2$\farcs pixels using a nearest-neighbor algorithm.  The nearest-neighbor
matching allows the map to be returned to a timestream in the same manner, 
but with the S/N improved by averaging over all hits on a given pixel.


%\Figure{l001_deconvolutionkernelcompare}{The effect of
%deconvolution on the iterative process can be seen in its effect on
%the residuals. From left to right: map, noise, smoothmap, model. Top
%to bottom kernel size: 14.4\arcsec, 21.6\arcsec, 31.2\arcsec,
%7.2\arcsec. The final version of the pipeline uses 14.4\arcsec, which
%has the result of leaving no flux in the residual at the location of
%Sgr B2, and does not ``dig a hole'' in the residual map, as the
%7.2\arcsec\ kernel does.}{fig:Deconvolution}{1.0}

\section{Noise and Systematic Effects}
\label{sec:Noise}

The error maps produced by the iterative mapping provide a natural way
of estimating the systematic error resulting from imperfect
subtraction of bright sources.  This results in ``ghosts'' of the
bright sources in the error maps $E$.  

% this is not true.  I can't do jackknifing....
% To estimate the size of this
% residual error, we also construct jackknife maps for each field, in
% which the astrophysical signal is made to exactly cancel.  This
% provides an estimate of the noise-only RMS in the maps, to which
% deviations from the ``ghosts'' may be compared.

The noise is correlated between adjacent pixels.  

The final per-pixel error estimate is given as ...

Because observing conditions varied widely during the survey, the RMS
noise levels obtained in the various fields of the BGPS also varies.  

\Figure{survey_depth}{The depth of the BGPS in the first quadrant
as a function of Galactic longitude.  Black is the standard deviation
of $0.9\arcdeg \times 0.9\arcdeg$ degree blocks centered on $b=0$ and
the longitude indicated; red is similar, but the estimator is
median(1/sqrt(weight map).}{fig:Depth}{1.0}

The depth can be converted into an estimate of mass via standard
estimates.

\section{Photometry}

To do accurate and meaningful photometry on the Bolocam maps requires
a good estimate of the noise (Section \ref{sec:Noise}) but also an
understanding of the spatial filtering imposed by the observing
strategy and the cleaning and mapping of the data.

The Bolocam maps produced are the convolution of the (nearly Gaussian)
primary beam $G(\xad)$, together with an effective high-pass
spatial filter $F(\xad)$,
\[
M'(\xad) = (G(\xad) \otimes F(\xad)) \otimes M(\xad)
\]
The effect of the high pass filter is that the effective PSF $B=G
\otimes F$ has the non-intuitive property that its area is {\em zero}:
\[
\int{B(\Omega) d\Omega} = 0
\]
Clearly, this complicates the definition of surface brightness, and
also imposes a limit on the size of features for which photometry may
be meaningfully performed.

\Figure{aper_phot_hpf_beam}{The effect of performing aperture
photometry on a point source of unit flux observed with the effective
Bolocam PSF $B$.  The black curve shows the area (measured in pixels)
as a function of the aperture radius for the filter measured via
simulation.  This is compared to the equivalent case for a Gaussian
beam (green).  The vertical lines show radii of $n/2 \times FWHM$ for
$n=1,2,3$.  One can see that the Gaussian beam has nearly reached a
constant area of $2 \pi \sigma^2$ by a radius of $3/2 \times FWHM$, at
which point any larger aperture would recover the full flux of the
source.  The actual Bolocam beam reaches its maximum effective area
sooner, and then drops precipitously as the size of the aperture is
increased.  }{fig:ESF}{1.0}

%\Figure{esf}{The effective spatial filter applied to the Bolocam
%maps due to atmospheric cleaning.}{fig:ESF}{1.0}

\clearpage 

\Figure{pca_comparison}{The fractional flux lost as a function of
source size for well-separated Gaussian sources with the FWHM as
indicated. Legend: number of PCA components }{fig:PCA_filter}{1.0}

\clearpage 

\Figure{deconvolution_comparison}{The fractional flux lost as a function of
source size for well-separated Gaussian sources with the FWHM as
indicated.  Legend: Size of deconvolution kernel used to make model }{fig:deconv_filter}{1.0}

\section{Final Maps}

The final maps are produced by coadding all observations of a single
square degree centered on a particular Galactic coordinate
$(l_0,b_0)$.  The maps are made in Galactic coordinates using a plate
carr\'{e} (FITS header CAR) projection.  As these maps are near the
coordinate system equator, the difference between truly equiareal
pixels and the pixels used is at most X\%.  The pixel size is
7.2\arcsec, chosen to be much smaller than the Bolocam beam, and
producing data squares 500 pixels on a side.

Noise maps are produced for each map, containing ...

\section{Public Data Release}

All processed maps are made available through
IPAC\footnote{http://irsa.ipac.caltech.edu/Missions/bolocam.html}.

%\citet{moore07}
%\citet{zinchenko97}

\section{Summary}

% first millimeter....
The BGPS is the first (sub)millimeter survey of a substantial fraction
of the Galactic Plane, providing an unbiased look at the high-density
gas most intimately associated with the earliest phases of star
formation.

\acknowledgments

We would like to acknowledge the staff and day crew of the CSO for their
assistance. The BGPS project is supported by the National Science Foundation
through NSF grant AST-0708403. J.A. was supported by a Jansky Fellowship from
the National Radio Astronomy Observatory (NRAO). The first observing runs for
BGPS were supported by travel funds provided by NRAO. Support for the
development of Bolocam was provided by NSF grants AST-9980846 and AST-0206158.
Team support was provided in part by NSF grant AST-0607793 to the
University of Texas at Austin.

We recognize and acknowledge the cultural role and reverence that the
summit of Mauna Kea has within the Hawaiian community. We are
fortunate to conduct observations from this mountain.

{\it Facilities:} \facility{CSO (Bolocam)}

\appendix

\section{Pointing Calculation}
\label{sec:PointingCalculation}

To calculate the position at which to point the telescope, the CSO
antenna computer computes an alt/az derived from the RA/Dec J2000
(heliocentric) catalog position.  It performs the following
calculation:

\begin{enumerate}

\item Precess catalog coordinates to current epoch.  The CSO by
default stores catalog positions in B1950.

\item Add aberration and nutation corrections.  This is the
``requested apparent'' RA and Dec reported by the antenna computer.

\item The equatorial coordinates are transformed to horizon
coordinates using the current local apparent sidereal time.

\item A 11-term model (``$C$-terms'') based on the optical pointing of
the telescope is then applied to $A$ and $Z$ along with $t$-terms (?),
the tilt of the alidade, and a refraction correction to obtain the
necessary mechanical pointing of the telescope to acquire the source.

\item The telescope then attempts to follow this track; the difference
between the requested $A$,$Z$ and the actual, measured angles of the
encoders (a result of servo errors) are also reported.

% are c4 and c5 different from the alidade_x,y tilt that are in the RPC file?

%char	32   pointing_file_name
%double	11   pointing_constant
%double	6    t_term_constant
%double	1    alidade_x_tilt
%double	1    alidade_y_tilt
%double	1    azimuth_t_term_offset
%double	1    elevation_t_term_offset
%double	1    tropospheric_refraction

\end{enumerate}

The CSO antenna computer reports this pointing information, including
geocentric RA/Dec, at a rate of 100 Hz.  This time series, along with
the field rotator encoder information from Bolocam is merged with the
bolometer time series and aligned.

To calculate the pointing used in the map from the telescope inputs,
the Bolocam software does the following:

\begin{enumerate}

\item Start with the reported $\alpha',\delta'$ from the CSO.  Remove
aberration and nutation corrections.  Precess from current epoch to
J2000.  This gives $\alpha,\delta$.

\item Apply the $\Delta A,\Delta E$ terms to $\alpha,\delta$ to
account for the difference between commanded and actual positions of
the telescope.

\end{enumerate}

After this is done, the remaining 


\section{Table of Absolute and In-Plane Pointing Sources}

\Table
{llrrrrc}
{Absolute Pointing Calibrators}
{Number & Alias & RA(J2000) & Dec(J2000) & l & b  & Flux (Jy)}
{tab:PCAL}
{
         1 &   1622-253 & 16 25 46.9 & -25 27 38.3 & 352.14 &  16.32 &  0.18 \\
         2 & 16293-2422 & 16 32 22.9 & -24 28 35.6 & 353.94 &  15.84 &  8.40 \\
         3 &   1657-261 & 17 00 53.2 & -26 10 51.7 & 356.70 &   9.75 &  0.07 \\
         4 &   NGC6334I & 17 20 53.4 & -35 47  1.7 & 351.42 &   0.64 &  2.50 \\
         5 &    NRAO530 & 17 33 02.7 & -13 04 49.5 &  12.03 &  10.81 &  2.70 \\
         6 &   1741-038 & 17 43 58.8 &  -3 50  4.6 &  21.59 &  13.13 &  1.44 \\
         7 &   1749+096 & 17 51 32.8 &   9 39  0.7 &  34.92 &  17.65 &  2.70 \\
         8 &      G5.89 & 18 00 30.4 & -24 04  0.5 &   5.89 &  -0.39 &  3.57 \\
         9 &        M8E & 18 04 53.0 & -24 26 39.4 &   6.05 &  -1.45 &  3.92 \\
        10 &     G10.62 & 18 10 28.7 & -19 55 49.8 &  10.62 &  -0.38 & 17.20 \\
        11 &   1830-211 & 18 33 39.9 & -21 03 40.0 &  12.17 &  -5.71 &  1.70 \\
        12 &      G34.3 & 18 53 18.6 &   1 14 58.3 &  34.26 &   0.15 & 31.30 \\
        13 &   1908-202 & 19 11 09.6 & -20 06 55.0 &  16.87 & -13.22 &  1.00 \\
        14 &      G45.1 & 19 13 22.1 &  10 50 53.4 &  45.07 &   0.13 &  3.70 \\
        15 &      OV236 & 19 24 51.0 & -29 14 29.8 &   9.34 & -19.61 &  8.60 \\
        16 &   1923+210 & 19 25 59.6 &  21 06 26.1 &  55.56 &   2.26 &  0.32 \\
        17 &   1954+513 & 19 55 42.7 &  51 31 48.6 &  85.30 &  11.76 &  0.60 \\
        18 &       CYGA & 19 59 28.5 &  40 44  1.7 &  76.19 &   5.75 &  0.25 \\
        19 &      K3-50 & 20 01 45.7 &  33 32 43.5 &  70.29 &   1.60 &  8.45 \\
        20 &   2005+403 & 20 07 44.9 &  40 29 48.6 &  76.82 &   4.30 &  0.40 \\
        21 &        ON1 & 20 10 09.1 &  31 31 37.7 &  69.54 &  -0.98 &  2.90 \\
        22 &    2013+37 & 20 15 28.7 &  37 10 59.6 &  74.87 &   1.22 &  1.40 \\
        23 &   2021+317 & 20 23 19.0 &  31 53  2.4 &  71.40 &  -3.10 &  0.54 \\
        24 &   2023+336 & 20 25 10.8 &  33 43  0.3 &  73.13 &  -2.37 &  0.92 \\
        25 &     GL2591 & 20 29 24.7 &  40 11 18.9 &  78.89 &   0.71 &  2.07 \\
        26 &    MWC 349 & 20 32 45.5 &  40 39 36.7 &  79.64 &   0.47 &  1.67 \\
        27 &       W75N & 20 38 36.4 &  42 37 34.5 &  81.87 &   0.78 &  4.46 \\
        28 &      3C418 & 20 38 37.0 &  51 19 12.7 &  88.81 &   6.04 &  0.62 \\
        29 &   CRL 2688 & 21 02 18.8 &  36 41 37.7 &  80.17 &  -6.50 &  1.90 \\
        30 &    NGC7027 & 21 07 01.6 &  42 14 10.2 &  84.93 &  -3.50 &  9.27 \\
        31 &   2201+315 & 22 03 15.0 &  31 45 38.4 &  85.96 & -18.78 &  0.35 \\
}

\Table
{llrrrrc}
{In-plane Millimeter Sources}
{Number & Alias & RA(J2000) & Dec(J2000) & l & b  & Flux (mJy)}
{tab:PPS}
{
   l351pps &            & 17 23 50.2 & -36 38 58.8 & 351.04 &  -0.34 &  -999.0 \\
   l354pps &            & 18 00 50.0 & -23 20 33.0 &   6.55 &  -0.10 &  -999.0 \\
   l357pps &            & 17 40 57.3 & -31 10 48.0 & 357.56 &  -0.32 &  -999.0 \\
   l000pps &            & 17 46 37.1 & -29 10 21.5 & 359.91 &  -0.31 &  -999.0 \\
  l000pps2 &            & 17 47 10.3 & -28 46 11.8 &   0.32 &  -0.20 &  -999.0 \\
   l002pps &            & 17 50 36.3 & -27 06  2.8 &   2.14 &   0.01 &  -999.0 \\
   l006pps &            & 17 57 34.4 & -23 58 12.0 &   5.64 &   0.24 &   989.0 \\
   l009pps &            & 18 06 14.2 & -20 31 58.0 &   9.61 &   0.19 &  1187.0 \\
  l009pps2 &            & 18 06 15.5 & -20 32  8.0 &   9.61 &   0.19 &  1189.0 \\
   l012pps &            & 18 10 50.5 & -17 55 54.0 &  12.42 &   0.51 &  1205.0 \\
   l015pps &            & 18 14 35.3 & -16 45 45.0 &  13.87 &   0.28 &  1085.0 \\
  l015pps2 &            & 18 21 08.5 & -14 31 48.0 &  16.58 &  -0.05 &   748.0 \\
   l018pps &            & 18 25 42.0 & -13 10 13.0 &  18.30 &  -0.39 &  1544.0 \\
   l020pps &            & 18 28 09.9 & -11 28 47.0 &  20.08 &  -0.13 &  1650.0 \\
   l021pps &            & 18 30 34.0 &  -9 34 54.0 &  22.04 &   0.22 &   993.0 \\
   l023pps &            & 18 34 54.6 &  -8 49 18.0 &  23.20 &  -0.38 &  1681.0 \\
   l024pps &            & 18 36 05.1 &  -7 31 21.0 &  24.49 &  -0.04 &  1851.0 \\
   l025pps &            & 18 36 16.7 &  -6 43  8.7 &  25.23 &   0.29 &  -999.0 \\
   l026pps &            & 18 37 18.4 &  -6 38 38.4 &  25.41 &   0.10 &  -999.0 \\
  l026pps2 &            & 18 39 04.9 &  -6 24 23.6 &  25.82 &  -0.18 &  -999.0 \\
   l027pps &            & 18 41 51.1 &  -5 01 48.0 &  27.36 &  -0.17 &  3478.0 \\
   l028pps &            & 18 44 19.0 &  -4 40 52.0 &  27.95 &  -0.55 &  -999.0 \\
  l028pps2 &            & 18 42 51.8 &  -4 00  1.9 &  28.39 &   0.08 &  -999.0 \\
   l029pps &            & 18 42 15.6 &  -3 34 44.5 &  28.70 &   0.41 &  -999.0 \\
  l028pps2 &            & 18 42 50.8 &  -3 59 46.1 &  28.40 &   0.09 &  -999.0 \\
   l033pps &            & 18 52 25.1 &   0 14 53.0 &  33.26 &  -0.11 &  -999.0 \\
   l035pps &            & 18 53 39.1 &   1 50 32.0 &  34.82 &   0.35 &   756.0 \\
   l037pps &            & 18 59 10.0 &   4 12 10.0 &  37.55 &   0.20 &  1045.0 \\
   l038pps &            & 19 01 53.2 &   4 12 51.0 &  37.87 &  -0.40 &  1487.0 \\
   l040pps &            & 19 05 40.9 &   6 26  5.0 &  40.28 &  -0.22 &  1700.0 \\
   l042pps &            & 19 10 33.5 &   9 08  7.0 &  43.23 &  -0.05 &  1247.0 \\
   l044pps &            & 19 11 54.4 &   9 35 53.0 &  43.80 &  -0.13 &  1829.0 \\
  l044pps2 &            & 19 13 28.4 &  10 53 41.0 &  45.12 &   0.13 &  1999.0 \\
   l048pps &            & 19 23 10.8 &  14 26 31.0 &  49.37 &  -0.30 &  1903.0 \\
   l050pps &            & 19 23 11.1 &  14 26 34.0 &  49.37 &  -0.30 &  2088.0 \\
   l079pps &            & 20 29 25.3 &  40 11 28.0 &  78.89 &   0.71 &  1390.0 \\
   l079pps &            & 20 29 25.2 &  40 11 30.0 &  78.89 &   0.71 &  2014.0 \\
  l079pps2 &            & 20 31 13.3 &  40 03 33.0 &  78.99 &   0.35 &   790.0 \\
   l080pps &            & 20 34 42.8 &  39 44 45.0 &  79.13 &  -0.37 &  1104.0 \\
  l080pps2 &            & 20 40 05.3 &  41 31 48.0 &  81.17 &  -0.11 &   961.0 \\
   l080pps &            & 20 30 28.7 &  41 16  6.0 &  79.88 &   1.18 &  1024.0 \\
   l082pps &            & 20 40 26.8 &  41 56 54.0 &  81.54 &   0.10 &   805.0 \\
l110.5npps &            & 23 05 10.7 &  60 14 50.6 & 110.11 &   0.05 &  -999.0 \\
   l111pps &            & 23 15 31.5 &  61 07 39.7 & 111.62 &   0.38 &  -999.0 \\
    l134p1 &            & 02 29 02.8 &  61 33 31.2 & 134.28 &   0.86 &  -999.0 \\
  l134p1-2 &            & 02 27 06.9 &  61 52 20.8 & 133.95 &   1.07 &  -999.0 \\
  l134p1-3 &            & 02 27 02.4 &  61 52 32.9 & 133.94 &   1.07 &  -999.0 \\
    l135p1 &            & 02 34 45.1 &  61 46 23.0 & 134.83 &   1.31 &  -999.0 \\
   l136p15 &            & 02 50 08.4 &  61 59 56.7 & 136.38 &   2.27 &  -999.0 \\
   l137p15 &            & 02 29 03.2 &  60 43 26.9 & 134.59 &   0.08 &  -999.0 \\
    IC1396 &            & 21 40 11.5 &  58 16 11.7 &  99.93 &   4.21 &  -999.0 \\
  IC1396-2 &            & 21 36 36.8 &  57 30 50.0 &  99.07 &   3.97 &  -999.0 \\
  IC1396-3 &            & 21 35 38.4 &  57 26 40.5 &  98.93 &   4.00 &  -999.0 \\
   l189pps &            & 06 08 50.1 &  21 38 19.0 & 188.94 &   0.87 &  -999.0 \\
 l189pps-2 &            & 06 08 49.7 &  21 38  5.7 & 188.95 &   0.87 &  -999.0 \\
   l192pps &            & 06 12 51.5 &  18 00 39.3 & 192.58 &  -0.05 &  -999.0 \\
 l192pps-2 &            & 06 12 51.5 &  17 59 36.3 & 192.59 &  -0.05 &  -999.0 \\
 l192pps-3 &            & 06 12 50.1 &  18 00 35.8 & 192.58 &  -0.05 &  -999.0 \\
 l192pps-4 &            & 06 12 50.3 &  17 59 28.8 & 192.59 &  -0.06 &  -999.0 \\
 l192pps-5 &            & 06 07 43.9 &  20 39 29.6 & 189.67 &   0.17 &  -999.0 \\
}

\section{Calculation of Color Corrections}
\label{app:ColorCorrections}

If an experiment has finite bandwidth ($t(\nu) \ne \delta(\nu -
\nu_c)$), to report a source surface brightness at a single
frequency, one must assume a source spectrum.  The power detected from
that source is assumed to be
\begin{equation}
P_{in} = \eta \; A \Omega \; \int I_0(\nu) t(\nu) d \nu
\end{equation}
(Here $\eta$ and $A \Omega$ are the optical efficiency and throughput
of the instrument, $I_0(\nu)$ is the nominal (assumed) surface
brightness of the source, and $t(\nu)$ is the bandpass transmission
normalized to $1.0$ at its peak.)  The effective band center $\nu_c$
is usually chosen such that
\begin{equation}
I_0(\nu_c) \simeq \frac{\int I_0(\nu) t(\nu) d \nu}
{\int t(\nu) d \nu} 
\end{equation}
The band centers for TopHat are calculated assuming a Rayleigh-Jeans
(RJ) source spectrum
\begin{eqnarray}
I_0(\nu_c) & = & I_{RJ}(\nu_c) \\
\nonumber & = & \tau \; 2 k T \; \frac{\nu_c^2}{c^2},
\end{eqnarray}
where $\tau$ is the optical depth of the source and $k$ is Boltzmann's
constant.  Since the detected power is assumed to be
\begin{equation}
P_{in} = \eta \; A \Omega \; \int \tau \; 2 k T \; \frac{\nu^2}{c^2}
\; t(\nu) d \nu,
\end{equation}
we can write
\begin{equation}
I_0(\nu_c) = \frac{P_{in}}{\eta \; A \Omega \int \nu^2 \; t(\nu) d \nu} \; \nu_c^2
\end{equation}
Now if we assume a different source spectrum, for example a greybody
with power-law emissivity, the assumed input power is
\begin{eqnarray}
P_{in} & = & \eta \; A \Omega \; \int I_{GB}(\nu) t(\nu) d \nu \\
\nonumber & = & \eta \; A \Omega \; \int \tau(\nu_0) (\nu / \nu_0)^\alpha B_\nu(T) t(\nu) d \nu
\end{eqnarray}
and the source spectrum inferred from the detected power is
\begin{eqnarray}
I_{GB}(\nu_c) & = & \tau(\nu_0) (\nu_c / \nu_0)^\alpha B_{\nu_c}(T) \\
\nonumber & = & \frac{P_{in}}{\eta \; A \Omega \int \nu^\alpha B_\nu(T) t(\nu) d \nu} \; \nu_c^\alpha B_{\nu_c}(T) \\
\nonumber & = & \frac{\int \nu^2 t(\nu) d \nu}{\int \nu^\alpha B_\nu(T) t(\nu) d \nu} \; \nu_c^{\alpha-2} B_{\nu_c}(T) \; I_{RJ}(\nu_c) \\
\nonumber & \equiv & \frac{I_{RJ}(\nu_c)}{K}.
\end{eqnarray}
This defines the color correction $K$ to apply to the reported TopHat
flux from a source if the source is assumed to have a greybody
spectrum with power-law emissivity.

We make a similar calculation for DIRBE, for which the band centers are 
computed assuming a spectrum with $\nu I(\nu)$ constant.  In this case, 
the correction is given by 
\begin{eqnarray}
K &=& \frac{\nu_c^{-1}}{\int \nu^{-1} t(\nu) d \nu} \left[\frac{\tau(\nu_0) (\nu_c / \nu_0)^\alpha B_{\nu_c}(T)}{\int \tau(\nu_0) (\nu / \nu_0)^\alpha B_{\nu}(T) t(\nu) d \nu} \right]^{-1} \\
\nonumber &=& \frac{\int \nu^\alpha B_{\nu}(T) t(\nu) d \nu}{\int \nu^{-1} t(\nu) d \nu} \; \nu_c^{-(\alpha+1)} B_{\nu_c}(T)
\end{eqnarray}
For an arbitrary experiment with bandpass $t(\nu)$ that reports its
surface brightness measurements assuming a spectrum $I_0(\nu)$, the
surface brightness assuming a different source spectrum $I_1(\nu)$ is
given by
\begin{eqnarray}
I_1(\nu_c) &=& I_0(\nu_c) / K \\
\nonumber &=& I_0(\nu_c) \frac{I_1(\nu_c)}{\int I_1(\nu) t(\nu) d \nu} \left[\frac{I_0(\nu_c)}{\int I_0(\nu) t(\nu) d \nu} \right]^{-1}
\end{eqnarray}

\bibliography{milkyway}

\end{document}







